% \documentclass[11pt]{asaproc}  
\documentclass[11pt]{article}  
\usepackage{graphicx}

\usepackage{times}
\usepackage{listings}
\usepackage{amssymb}

\setlength{\parindent}{0in}
\setlength{\parskip}{0.1in}

\title{A Method for Avoiding Data Disclosure While
Automatically Preserving Multivariate Relations}
\author{
Norman Matloff\thanks{Dept. of Computer Science, University of
California, Davis}
\and
Patrick Tendick\thanks{Avaya}
}

\begin{document}

\maketitle

\begin{abstract}

\noindent 
Statistical disclosure limitation (SDL) methods aim to provide analysts
general access to a data set while limiting the risk of disclosure of
individual records.  Many methods in the existing literature are 
aimed only at the case of univariate distributions, but the 
multivariate case is crucial, since most statistical analyses are 
multivariate in nature.  Yet preserving the multivariate structure of 
the data can be challenging, especially when both continuous and
categorical variables are present.  Here we present a new SDL method 
that automatically attains the correct multivariate structure,
regardless of whether the data are continuous, categorical or mixed.  In
addition, operational methods for assessing data quality and risk will
be explored.

\end{abstract}

\section{Introduction}

Statistical disclosure limitation (SDL) methods aim to provide analysts
general access to a data set while limiting the risk of disclosure of
individual records.  Common methods include noise addition, swapping of
parts of records, replacing data by synthetic equivalents, suppression
of small cells in contingency tables, and so on \cite{duncan}. 

% (Duncan, 2011)..

Long the field of statistical research, in recent years SDL issues have
attracted the interest of computer scientists \cite{dwork}.  There has
been a marked contrast in the approaches taken by the two communities:
The statistical view is that of serving research analysts who wish to do
classical inference from samples, while the computer scientists, coming
from a cryptographic background, have viewed the data itself as the
primary focus.  In other words, in the computer science approach, the
`S' in SDL has perhaps had lesser attention, compared to the
statisticians' view of things.  However, there is some indication of
increasing interaction between the two groups \cite{abowd} \cite{duchi}.

For an overview of how methodology has been refined and expanded over
time, compare a 1989 survey paper \cite{adam}, a 2002 Census Bureau
viewpoint \cite{census2002}, the current statistical view \cite{duncan},
and the more recent computer science approach \cite{dwork}.
% \cite{winkler}.

Whatever approach is taken, a primary goal remains statistical
analysis by the end user.  And in order to perform meaningful
statistical analysis on the data, {\bf one's methods must at least
approximately preserve multivariate structure}.  Most ststistical
analysis --- linear regression, logistic models, princple components
analysis, the log-linear model and so on -- are inherently multivariate.
Unfortunately, many existing SDL methods place little or no emphasis
on this aspect, and this is an absolutely central issue.  
Regression coefficient estimates, for instance, can turn out
substantially biased as a result.  As noted in \cite{nivule},

\begin{quote}
...[in using] noise addition techniques...the original data
suffers loss of some of its statistical properties even while
confidentiality is granted, thus making the dataset almost meaningless
to the user of the published dataset. 
\end{quote}

The above statement applies only to independent noise variables. Noise
addition methods can preserve the multivariate structure of continuous
variables, if the data come from an approximate multivariate normal
distribution, by adding correlated noise \cite{matloff1986}
\cite{kim} \cite{tendick}.  However, this does not apply to the
discrete-variable case, and moreover, the same problems apply to most if
not all of the other major classes of SDL methods.

Developing methodology for the mixed continuous/discrete case is a
difficult problem; see \cite{manrique} and the citations therein for
some existing methodology.  To broaden the methods available to Data
Stewardship Organizations (DSOs), a new method is proposed in this paper
to deal with the multivariate structure preservation problem. Our method
has several important advantages:

\begin{itemize}

\item The method works on general data, i.e.\ continuous, discrete or
mixed. 

\item The method does not require the DSO to estimate the dependency
structure between the variables, or make assumptions regarding that
structure.

\item The method has several tuning parameters, affording database
administrator broad flexibility in attaining the desired balance between
privacy and statistical usability.

\end{itemize}

\section{Overview of the Method}
\label{overview}

Let $W_{ij}, i = 1,...n, j = 1,...,p$ denote our original data on $n$
individuals and $p$ variables.  Choose $\epsilon > 0$ and $0 < q \leq
1$.  Then we form our released data $W'_{ij}$ as follows:

For $i = 1,...n$:

\begin{itemize}

\item Consider record $i$ in the data base:

\begin{equation}
r_i = (W_{i1},...,W_{ip})
\end{equation}

\item With probability $1-q$, skip the next steps.

\item Find the set $S$ of points in the data set within $\epsilon$
distance of $r_i$.

\item Draw a random sample (\underline{with} replacement) of $p$ items
from $S$, resulting in values $a_{km}, k = 1,...,p, m = 1,...,p$.

\item For $j = 1,...,p$, set 

\begin{equation}
W'_{ij} = a_{jj}
\end{equation}

\end{itemize}

\section{Theoretical Justification}

{\bf Theorem:}  Consider a bivariate random vector $(X,Y)$ and $\epsilon >
0$.  For any $t$ in $R^2$, let $A_{t,\epsilon}$ denote the $\epsilon$
neighborhood of $t$, defined by some metric.  Let $F$ denote the
cdf of $(X,Y)$, and define $G_{t,\epsilon}$ to be the
conditional cdf of $(X,Y)$, given that that vector is in
$A_{t,\epsilon}$.  Finally, given $(X,Y)$, define {\it independent}
random variables $U$ and $V$ to be drawn randomly from the first- and
second-coordinate marginal distributions of $G_{(X,Y),\epsilon}$,
respectively.  Then

\begin{equation}
\lim_{\epsilon \rightarrow 0}
P
\left (
U \leq a \textrm{ and } V \leq b
\right )
= F(a,b)
\end{equation}

for all $-\infty < a,b < \infty$.  

In other words, as $\epsilon$ goes to 0, the distribution of $(U,V)$
goes to that of $(X,Y)$, {\it even though $U$ and $V$ are conditionally
independent}.

{\bf Proof:}

Given $(X,Y) = t = (t_1,t_2)$,

\begin{equation}
\lim_{\epsilon \rightarrow 0} U = t_1
\end{equation}

and

\begin{equation}
\lim_{\epsilon \rightarrow 0} V = t_2
\end{equation}


Then by bounded convergence,

\begin{eqnarray}
\lim_{\epsilon \rightarrow 0}
P
\left (
U \leq a \textrm{ and } V \leq b
\right )
&=& 
\lim_{\epsilon \rightarrow 0}
E \left [
P
\left (
U \leq a \textrm{ and } V \leq b
~|~ X,Y \right )
\right ] \\ 
&=& 
\lim_{\epsilon \rightarrow 0}
E \left [
P(U \leq a ~|~ X,Y ) \cdot
P(V \leq b ~|~ X,Y ) 
\right ] \\
&=& E \left [
1_{X \leq a} \cdot
1_{Y \leq b}
\right ] \\
&=& E \left [
1_{X \leq a \textrm{ and } Y \leq b}
\right ] \\
&=& P(U \leq a \textrm{ and } V \leq b) \\
&=& F(a,b)
\end{eqnarray}

$\blacksquare$

\bigskip

The key word {\it independent} in the above theorem has a major
implication:  We can make our released data approximate the multivariate
distribution of the original data (or the population from which the
latter are drawn), {\bf without knowing or even estimating the
multivariate relationship of our variables}.  We simply sample {\it
independently} from $S$, yet attain the correct {\it dependency}
relationship among the variables.

The bit of seeming similarity between this new method and data swapping
is largely deceiving.  Clearly our method does do swapping of values,
and in some sense our neighborhood approach relates somewhat to the fact
that data swapping is typically conducted on a within-stratum basis,
such as strata defined by age and race; a stratum then has some
similarity to our neighborhoods. 

But actually the two methods are quite different.  First, with data
swapping, records from one stratum are switched with those in {\it
another} stratum, whereas in our method everything stays within the same
neighborhood.  
% This is very important, because with data swapping,
% choosing the stratifying variables precludes analysts doing statistical
% analyses that include those variables.
Moreover, 
our neighborhoods can grow or shrink in size, as opposed to
the fixed stratum size in data swapping.  

\section{Code and Tuning Parameters}

The method provides the DSO with excellent flexibility in achieving the
desired balance between privacy and accurate multivariate structure, via
the following tuning tuning parameters:

\begin{itemize}

\item The neighborhood radius, $\epsilon$.

\item The distance metric.

\item The proportion of modified records.  

\end{itemize}

Code implementing the method is provided on GitHub ({\it
https://github.com/matloff/statdb}) to implement the method.  The call
form is

\begin{lstlisting}
nbrs(z, eps, modprop = 1, wts = NULL) 
\end{lstlisting}

where {\bf eps} is $\epsilon$, {\bf modprop} is $q$ in the algorithm
in Section \ref{overview}, and the {\bf wts} argument controls the
distance metric, to be explained shortly.  The return value is the
released data set.

It is assumed that all categorical variables have been converted to
dummy variables.  Ordinary Euclidean distance is used on the scaled
data, including any dummy variables.  Scaling places all the variables
on the same footing --- all now have standard deviation 1 --- but there
is still a difference between the continuous variables and the dummies
and other discrete variables, as follows.

As sample size $n$ grows (treating the original data as a sample from
some population), one would want $\epsilon$ to become smaller, but this
would not work well for the discrete variables.  With large $n$, the
latter would come to dominate the distance metric, and one could not
drop $\epsilon$ below some minimum threshhold.  Thus the {\bf wts}
argument provides the DSO with a tool to reduce that dominance, by
allowing the weights of the discrete variables (or others) to decrease
as $n$ increases.

If for example we set {\bf wts = c(5,12,13,rep(0.6,3))}. then in
computing distances the variables in columns 5, 12 and 13 of the data
matrix are reduced in weight by a factor of 0.6.  

\section{Selection of Tuning Parameters}

In some modern statistical methods, the user is faced with selection of
a large number of tuning parameters, both numeric and policy-oriented,
such as in the SIS package \cite{fan}.  The user may find the task of
setting those parameters daunting and bewildering.

In SDL settings, though, the DSO may {\it welcome} the selection of
tuning parameters. Achieving a good balance between accuracy of the
released data and disclosure risk, so from the DSO's point of view, the
more tuning parameters the better.

For a given set of tuning parameters, the DSO wishes to assess

\begin{itemize}

\item [(a)]
whether the results of the analysis on the our released data set are
reasonbly close to those of the original data, and 

\item [(b)]
whether records that were at risk in the original data are masked 
sufficiently well in the released data.

\end{itemize}

In setting these parameters, the DSO must take into account not only the
desired balance between (a) and (b), but also the values of $n$ and $p$.
For fixed $p$, the larger $n$ is, the fewer the number of uniquely
identifiable individuals in the data, and thus the decreased need for
privacy measures.\footnote{As noted, we are treating the data as a
sample from some (tangible or conceptual) population.  As such, the
notion of a {\it population unique}, seen in some of the SDL literature,
doesn't apply.  If a combination of the categorical variables appears in
our data, then by definition that combination has nonzero probability in
the population, and we'll get more and more individuals of that type as
$n$ grows.  For continuous variables, a similar statement holds in the
sense that as $n$ grows, we will have more and more individuals near the
given value.} On the other hand, for fixed $n$, the larger the value of
$p$, the more potential identifiable uniques.

Regarding (a), we propose an operational approach.\footnote{We have not
seen this in the literature, though it is likely that some DSOs have
experimented with this approach.}  Though many authors have proposed
global measures of distance between the original and released data sets,
we suggest gauging the accuracy of the latter in a manner that is
motivted by the intended usag of the data, namely statistical analyses.

\section{Example}

We used the Census data set in the package {\bf regtools} ({\it
https://github.com/matloff/regtools}) to simulate an employee
database, sampling 5000 records from this data.\footnote{Since this is
just an illustration, the data were not cleaned, and some WageIncome
values were 0 that probably should have been designated as missing.}

The call used was

\begin{lstlisting}
> p1p <- nbrs(p1,eps=0.3,wts=c(2,4,5,rep(0.05,3)))
\end{lstlisting}

To gauge how close this new version of the data was to the original, we 
ran a linear regression analysis, predicting WageIncome from Age,
Gender, WeeksWorked, MSDegree and PhD.  The estimated coefficients
for the original and modified data were

% \begin{table}
% \begin{center}
% \vskip 0.5in
\begin{tabular}{|r|r|r|r|r|}
\hline
Age & Gender & WeeksWorked & MS & PhD \\ \hline 
447.2 & -9591.7 & 1286.4 & 17333.0 & 21291.3 \\ \hline 
518.8 & -8695.9 & 1300.9 & 17686.6 & 21953.4 \\ \hline 
\end{tabular}
% \end{center}
% \caption{}
% \label{eps1}
% \end{table}

The results were quite good for the last three coefficients, and fairly
good for the first two.

In the original data set, there was one female worker with age under 31L

\begin{lstlisting}
> p1pc[p1pc$sex==2 & p1pc$phd==1 & p1pc$age < 31,]
          age sex wkswrkd ms phd wageinc
7997 30.79517   2      52  0   1  100000
\end{lstlisting}

How well was she hidden in the modified data?  Quite well, it turns out:

\begin{lstlisting}
> p1pc[p1pc$sex==2 & p1pc$phd==1 & p1pc$age < 31,]
           age sex wkswrkd ms phd wageinc
6240  25.22471   2       0  0   1       0
6149  28.27956   2       0  0   1       0
13485 26.42564   2      52  0   1   32000
893   28.33435   2      52  0   1   49900
\end{lstlisting}

There are now four women fitting the given conditions, none of which was
the one we highlighted in the original data, worker number
7997.\footnote{Of course, ID numbers would be suppressed.}

Next, we tried $\epsilon = 0.2$.  The new regression coefficients were


% \begin{table}
% \begin{center}
% \vskip 0.5in
\begin{tabular}{|r|r|r|r|r|}
\hline
Age & Gender & WeeksWorked & MS & PhD \\ \hline 
492.5 & -10266.6 & 1284.2 & 18187.0 & 21649.7 \\ \hline 
\end{tabular}
% \end{center}
% \caption{}
% \label{eps2}
% \end{table}

Now, however, there were no workers in the modified data set satisfying
the given conditions:

\begin{lstlisting}
> p1pc[p1pc$sex==2 & p1pc$phd==1 & p1pc$age < 31,]
[1] age     sex     wkswrkd ms      phd     wageinc
<0 rows> (or 0-length row.names)
\end{lstlisting}

This of course just barely scratches the surface of the various tuning
parameter values that the DSO could experiment with, in addition to
doing so on other types of analyses, say principle components analysis.


% library(regtools)
% data(prgeng)
% pe <- prgeng
% # dummies for MS, PhD
% pe$ms <- as.integer(pe$educ == 14)
% pe$phd <- as.integer(pe$educ == 16)
% # computer occupations only
% pecs <- pe[pe$occ >= 100 & pe$occ <= 109,]
% pecs1 <- pecs[,c(1,7,9,12,13,8)]
% set.seed(9999)
% p1 <- pecs1[sample(1:nrow(pecs1),1000),]
% table(p1$wkswrkd)
% set.seed(999999)
% p1p <- nbrs(p1,eps=0.3,wts=c(2,4,5,0.rep(2,3)))
% lm(wageinc ~ .,data=p1p)
% p1pc <- na.omit(p1p)
% p1pc[p1pc$wkswrkd == 43,]
% which(rownames(p1p) == 5016)
% p1p[120,]

\section{Discussion}

Note that ``a little bit of privacy can go a long way'':  As long as the
intruder knows that the data have been modified (even for the
nonsensitive variables), there may be enough doubt in his/her mind as to
make the data useless for nefarious purposes (while still being very
useful for legitimate purposes).

In databases with large $p$, one must take into account the Curse of
Dimensionality \cite{beyer}.  The DSO may choose to use a weighted
distance metric, with the weights going to 0 as the variable index goes
to infinity \cite{matloff2015}.

In general, the choice of $\epsilon$ must also be made carefully
This approach does require fairly large data sets, so that for example
the set $S$ contains some female workers.  One might even allow the
value of $\epsilon$ to vary from record to record.

\section{Work to Be Done}

The presentation here is of course preliminary, and many aspects need to
be explored.  The method will be tried on a wide variety of data sets;
effects of varying the tuning parameters will be explored; the possible
usefulness of making the values of the tuning parameters vary from one
record to another will be investigated; and so on.

\begin{thebibliography}{}

\bibitem{abowd} J. Abowd (2015). Comments as the Discussant in a session
at JSM 2015.

\bibitem{adam} 
N.R. Adam and J.C. Wortmann (1989).
Security-Control Methods for Statistical Databases: A Comparative Study,
{\it ACM Computing Surveys}, 21(1989).

\bibitem{beyer} K. Beyer, J. Goldstein, R. Ramakrishnan (1999).  When Is
``Nearest Neighbor'' Meaningful?, {\it Lecture Notes in Computer
Science}, Volume 1540, 1999, 217-235.

\bibitem{dwork} C. Dwork (2008).  Theory and Applications of Models of
Computation {\it Lecture Notes in Computer Science, Vol. 4978}, M.
Agrawal {\it et al} (es.), 1-19.

\bibitem{census2002} U.S. Census Bureau (2002). 
{\it Census Confidentiality and Privacy: 1790 - 2002},
{\it http://www.census.gov/prod/2003pubs/conmono2.pdf}.

\bibitem{duncan} Duncan, G., Elliot, M., Salazar, G. (2011).  {\it
Statistical Confidentiality: Principles and Practice}, Springer. 

\bibitem{fan} J. Fan, Y. Feng, D. Saldana, R. Samworth and Y. Wu (2015).
``Package `SIS''', CRAN, {\it
https://cran.r-project.org/web/packages/SIS/index.html},

\bibitem{kim} J. Kim (1986).  A Method for Limiting Disclosure in
Microdata Based on Random Noise and Transformation, {\it Proceedings of
ASA Section on Survey Research Methods}, 370-374.

\bibitem{manrique} Manrique-Vallier, D., Reiter, J. (2012).
``Estimating Identification Disclosure Risk Using Mixed Membership
Models,'' {\it JASA}, 107, 1385-1394.

\bibitem{matloff1986} N. Matloff (1986). Another Look at the Use of
Noise Addition for Database Security. {\it Proceedings of the 1986 IEEE
Symposium on Security and Privacy}, April 1986, pp. 173-180.

\bibitem{matloff2015}  N. Matloff (2015).  Big-n vs.\ Big-p in Big Data,
in {\it Handbook of Big Data}, Buhlmann and Kane (eds.), Chapman and
Hall, to appear.

\bibitem{nivule}
K. Mivule (2011).  {\it Utilizing Noise Addition for Data Privacy, an 
Overview}, {\it http://arxiv.org/pdf/1309.3958.pdf}.

\bibitem{shlomo}
Shlomo, N., Skinner, C. (2008).
``Assessing the Protection Provided by Misclassification-Based Disclosure
Limitation Methods for Survey Microdata,'' {\it Annals of Applied
Statistics}, 4,3, 1291-1310.

\bibitem{tendick} P. Tendick and N. Matloff (1994).  A Modified Random
Perturbation Method for Database Security, {\it ACM Transactions on
Database Systems}, 19, 47-63.

\bibitem{winkler} W. Winkler (2005). {\it Microdata Confidentiality
References}, {\it
https://www.census.gov/srd/sdc/Winkler.List.May.2005.pdf}.

\end{thebibliography}{}

\end{document}

